\documentclass[14pt]{extarticle}
\usepackage[utf8]{inputenc}
\usepackage{hyperref}
\usepackage{enumitem}
\usepackage{parskip}
\usepackage{setspace}
\usepackage{titlesec}
\usepackage{amsmath}
\usepackage[margin=0.5in]{geometry}
\usepackage{fancyhdr}

\title{\Huge Very Afraid}
\author{\Large M. John Harrison}
\date{\large January 27, 2007...10:14 am}

\pagestyle{fancy}
\fancyhf{}
\fancyfoot[C]{\thepage}
\renewcommand{\headrulewidth}{0pt}

\begin{document}

\maketitle

\vspace{-2em}
\begin{center}
  \normalsize Originally published on 
  \href{https://web.archive.org/web/20080410181840/http://uzwi.wordpress.com/2007/01/27/very-afraid/}{uzwi.wordpress.com}
\end{center}

Every moment of a science fiction story must represent the triumph of writing over worldbuilding.



Worldbuilding is dull. Worldbuilding literalises the urge to invent. Worldbuilding gives an unnecessary permission for acts of writing (indeed, for acts of reading). Worldbuilding numbs the reader’s ability to fulfil their part of the bargain, because it believes that it has to do everything around here if anything is going to get done.

Above all, worldbuilding is not technically necessary. It is the great clomping foot of nerdism. It is the attempt to exhaustively survey a place that isn’t there. A good writer would never try to do that, even with a place that \emph{is} there. It isn’t possible, \& if it was the results wouldn’t be readable: they would constitute not a book but the biggest library ever built, a hallowed place of dedication \& lifelong study. This gives us a clue to the psychological type of the worldbuilder \& the worldbuilder’s victim, \& makes us \emph{very afraid}.

\section*{THE FOLLOWING NOTES WERE ADDED ON DECEMBER 31, 2007:}

These observations will be of interest only to generic fantasy readers \& writers. They do not form an integrated piece. They were written as notes, or emails to other writers, in separate attempts to clarify my position. They contain repetitions \& restatements, \& while there is some steady movement towards a set of conclusions, I’ve made no attempt to turn them into an article. The element of provocation has been left in.

Responses to the original posts, mostly negative \& some more anxious than others, are numerous \& can be found by Googling “M John Harrison +Worldbuilding” or anything similar.

When I use the term “writing”, here or in the original posts, I am not referring to prose, but to every aspect of the process.

When I make a distinction between writers \& worldbuilders I am making a distinction not just between uses of a technique, but between suites of assumptions about language, representation \& the construction of “the” world as well as “a” world.

When I use the term “worldbuilding fiction” I refer to immersive fiction, in any medium, in which an attempt is made to rationalise the fiction by exhaustive grounding, or by making it “logical in its own terms”, so that it becomes less an act of imagination than the literalisation of one. Representational techniques are used to validate the invention, with the idea of providing a secondary creation for the reader to “inhabit”; but also, in a sense, as an excuse or alibi for the act of making things up, as if to legitimise an otherwise questionable activity. This kind of worldbuilding actually undercuts the best and most exciting aspects of fantastic fiction, subordinating the uncontrolled, the intuitive \& the authentically imaginative to the explicable; and replacing psychological, poetic \& emotional logic with the rationality of the fake.

I am aware that something describable as worldbuilding goes on in the representational genres, and in even the most minimal of “mundane” fictions; the strength of my position depends on that awareness. I see no technical distinction between the worldbuilding of the representational writer—the travel writer or memoirist—\& the worldbuilding of the fantasist. I have a certain amount of experience with both; \& a fair amount of experience of sailing back \& forth across the line between them. I agree to some extent with Aldous Huxley’s description of fantasy as “foreign travel of the imagination”. The distinction I would make between the two kinds of worldbuilding is in a sense Baudrillardian. But though I see fantasy worldbuilding as parasitic on its quotidian cousin, I also see it as not much more than a matter of the kidnapping \& abuse of some techniques which don’t, recently, have much dignity even in their proper place. It’s no big deal until you get behind it to the ideology. After that it becomes important but not in the context of writing fantasy fiction, see below, Notes 3.

\subsection*{Notes 1: Being \& Simulating}

Some of it is a matter of aesthetics. I think Katherine Mansfield could “build a world” in thirty words \& a couple of viewpoint changes, \& that Chekhov could cram more into four thousand words than Dickens got into three hundred thousand.

But much of it is a matter of ideology. The whole idea of worldbuilding is a bad idea about the world as much as it is a bad idea about fiction. It’s a secularised, narcissised version of the fundamentalist Christian view that the world’s a watch \& God’s the watchmaker. It reveals the bad old underpinnings of the humanist stance. It centralises the author, who hands down her mechanical toy to a complaisant audience (which rarely thinks to ask itself if language can deliver on any of the representational promises it is assumed to make), as a little god. And it flatters everyone further into the illusions of anthropocentric demiurgy which have already brought the real world to the edge of ecological disaster.

My feeling is that the reader performs most of the act of writing. A book spends a very short time being written into existence; it spends the rest of its life being read into existence. That’s why I find in many current uses of the term “active reading” such a deeply ironic tautology. Reading was always “active”; the text itself always demanded the reader’s interaction if the fiction was to be brought forth. There was always a game being played, between writers and readers (for that matter between oral storytellers \& listeners), who knew they were gaming a system, \& who were delighted to engage each other on those terms.

Worldbuilding is the province of people who, like Tolkien, actually resist the idea it’s a game, and have installed their “secondary creation” concept as an aggressive defense of that position.

The worst mistake a contemporary f/sf writer can make is to withhold or disrupt suspension of disbelief. The reader, it’s assumed, wants to receive the events in the text as seamless \& the text as unperformed. The claim is that nobody is being “told a story” here, let alone being sold a pup. Instead, an impeccably immersive experience is playing in the cinema of the head. This experience is somehow unmediated, or needs to present itself as such: any vestige of performativeness in the text dilutes the experience by reminding the reader that the “world” on offer is a rhetorical construct. All writing is a shell game, a sham: but genre writing mustn’t ever look as if it is. This seems to me to ignore the genuine sleight-of-hand pleasures of conjuring in favour of a belief in magic, a kind of non-writing which claims to be rather than to simulate.

\subsection*{Notes 2: Bandwidth}

I’m interested in how worldbuilders construct the real world. How do they describe the process of writing \& reading about it, for instance? Do they envisage writing as a kind of camera, which allows them to photograph London—or cheese—or a giraffe—and pass the picture to the reader, who then sees exactly what they saw? For that matter, would they describe photography itself as an objectively representational process? Perhaps they would, and perhaps that’s one of the main reasons why worldbuilding fantasy strikes one as so amazingly Victorian a form.

You cannot replicate the world in some symbols, only imply it or allude to it. Even if you could encode the world into language, the reader would not be able to decode with enough precision for the result to be anything but luck. (\& think how long it would take!) Writing isn’t that kind of transaction. Communication isn’t that kind of transaction. It’s meant to go along with pointing and works best in such forms as, “Pass me that chair. No, the green one.”

Writing does something else. It not only invites but relies upon reader-participation. Writing and reading are complementary aspects of the same process; much of what appears to be the work of writing is in fact done by the reader in the act of reading. While the writer takes advantage of this, making implications \& inviting the reader to do the rest, the worldbuilder—lonely \& godlike \& in control of (or attempting to be in control of) every piece of footage retrieved from her obsessive creation—induces dependency in the audience, then discovers in the subsequent delirious spiral of self-fulfilling prophecy an excuse to take even more responsibility out of their hands. God’s in her Heaven \& all’s right with the “world”.

It’s control-freakery on a scale that reminds you instantly of the other kind of worldbuilding—the political kind. That’s why I am “very afraid” of worldbuilders. They tend to be quite managing, even in real life.

\subsection*{Notes 3: It’s All Down Here in Black \& White}

The transaction we talk about when we talk about reading goes on not between the writer \& the reader but between the reader \& the text. The writer (as opposed to the worldbuilder) plans for this inevitability, presenting a spread of more or less “possible” interpretations tied to the themes \& meanings of the story, and allowing—or perhaps impishly not quite allowing—for the cultural library \& types of interpretative tool any given reader might bring to the text. In this view, any reading, of any kind of fiction, is emergent from the interaction of more variables than can be defined or consciously managed by either writer or reader. There seems to me little point trying to deny that this happens whether, as the writer, you encourage it or not. Any other view of the writing/reading process is at best idealistic \& at worst contains an appeal to telepathy (the idea that I can somehow pass my vision to you without mediation, the ultimate paradoxical utopia of the representational).

The writer—as opposed to the worldbuilder—must therefore rely on an audience which begins with the idea that reading is a game in itself. I don’t see this happening in worldbuilding fiction. When you read such obsessively-rationalised fiction you are not being invited to interpret, but to “see” and “share” a single world. As well as being based on a failure to understand the limitations of language as a communications tool (or indeed the limitations of a traditional idea of what communication can achieve), I think that kind of writing is patronising to the reader; and I’m surprised to find people talking about “actively reading” these texts when they seem to mean the very opposite of it. The issue is: do you receive—is it possible to receive—a fictional text as an operating manual? Or do you understand instead that your relationship with the very idea of text is already fraught with the most gameable difficulties \& undependabilities? The latter seems to me to be the ludic point of reading: anything else rather resembles the—purely functional—act of following instructions on how to operate a vacuum cleaner.

Since a novel is not an object of the same order as a vacuum cleaner, and since the “world” a worldbuilder claims to build does not in fact exist in the way a vacuum cleaner exists, why would you want to try \& operate it as if it was one?

In fact you wouldn’t, unless you were already experiencing confusion about what is functional \& what isn’t.

This aspect of the contemporary relationship between readers \& fiction is complicated further by the fact that, prior to any act of reading, we already live in a fantasy world constructed by advertising, branding, news media, politics and the built or prosthetic environment (in EO Wilson’s sense). The act of narcissistic fantasy represented by the wor(l)d “L’Oreal” already exists well upstream of any written or performed act of fantasy. JK Rowling \& JRR Tolkien have done well for themselves, but—be honest!—neither of them is anywhere near as successful at worldbuilding as the geniuses who devised “Coke”, or “The Catholic Church”. Along with the prosthetic environment itself, corporate ads \& branding exercises are the truly great, truly successful fantasies of our day. As a result the world we live in is already a “secondary creation”. It is already invented. Epic fantasies, gaming \& second lives don’t seem to me to be an alternative to this, much less an antidote: they seem to me to be a smallish contributory subset of it.

The piece that began all this, “What It Might Be Like To Live In Viriconium”, has been up at Fantastic Metropolis for at least five years, maybe longer. It was written in 1996, and originally published in a British print fanzine in 1997. The notes on which it was based were made as early as 1992. Since 1992, the feeling I had that this was an essentially political issue, \& not really much to do with epic fantasy or Tolkien movies or gaming in themselves, has only grown. That’s what I meant when I said at the end of my “licensed settings” post that I wouldn’t write Viriconium again, or write an article like “What It Might Be Like To Live In Viriconium”.

As we emerge from the trailing edge of postmodernism we begin to see how many of its by-now-naturalised assumptions need challenging if it isn’t to become as much of a dead hand as the modernism it revised into existence to be its opposite. The originally vertiginous and politically exciting notion of relativism that underlies the idea of “worlds” is now only one of the day-to-day huckstering mechanisms of neoliberalism. My argument isn’t really with writers, readers or gamers, (or even with franchisers in either the new or old media); it is a political argument, made even more urgent as a heavily-mediatised world moves from the prosthetic to the virtual, allowing the massively managed and flattered contemporary self to ignore the steady destruction of the actual world on which it depends. This situation needs to change, and it will. At the moment, the fossilised remains of the postmodern paradigm (which encourages us to believe three stupid things before breakfast: firstly that we can change the real world into a fully prosthetic environment without loss or effort; secondly that there are no facts, only competing stories about the world; \& thirdly that it’s possible to meaningfully write the words “a world” outside the domains of imagination or metaphor, a solecism which allows us to feel safely distant from the consequences of our actions) are in the way of that.

\section*{Comments}

\begin{itemize}[leftmargin=*]

\item \textbf{farah3}

January 28, 2007 at 12:36 pm

Surely the way to do worldbuilding well is to scrape away at the painted picture until only the tiniest flakes of the world you have built remains. And it’s from \emph{those} flakes that your readers see the world, as we indeed see our own.

A good example of what you are describing tho was Neal Stephenson’s trilogy. By the end I felt I would have far rather read his original source material

\item \textbf{uzwi}

January 28, 2007 at 1:47 pm

Difficult to answer this, Farah, since for me the process works the opposite way round. There’s rarely anything to scrape away. I layer it up. For me, three-decker fantasy worlds became obsolete the day Bob Dylan wrote “All Along the Watchtower”, \& got the worldbuilding effect in 130 words. For that, you can bet he didn’t “construct” a whole world then pare it down to a couple of perfect indices. But I do agree that less is more: indeed minimalism \& particularity are the best aids to successfully counterfeiting the unreal (or for that matter the real). As for “original source material”, the horror, the horror. Let’s not go there or I’ll get in more trouble.

\item \textbf{airship}

April 16, 2007 at 2:05 pm

Bullshit. Lord of the Rings is widely hailed as the best, most readable fantasy ever written. Why? Precisely because its author spent over thirty years building the world it is set in. He developed fully-functional languages and a multi-millenial history for Middle-Earth. Most of it isn’t even touched on in the trilogy, but you just know IT’S THERE. LotR would be out of print today if it weren’t for Tolkien’s worldbuilding.

\item \textbf{elfsternberg}

April 18, 2007 at 11:34 pm

I’m afraid I don’t find this observation very compelling: it’s a specific example of Twain’s “The episodes of a tale shall be necessary parts of the tale, and shall help develop it,” or Vonnegut’s advice on writing: “Use the time of a total stranger in such a way that he or she will not feel the time was wasted.”

You seem to be addressing more the issue of worldbuilding while the reader is watching, rather than worldbuilding in general. One of the lessons a new writer hears over and again is that he should keep character profiles. Your character’s actions should begin and end as expressions of his or her fundamentals. A character can be motivated from first principles; may have public and private faces; or may just be a collection of impulses bound together in a peculiar personality. But the character must be a whole person, someone with whom the audience can identify, if not empathize.

The world is a character. It has wants, most of which involve stymieing the reader. It has a personality, it has impulses, and it has principles. You don’t need to get everything absolutely right; you simply need to not get anything wrong. Sometimes, yes, that means that in the writer’s notes for a story or a series, he makes up and writes down details about the biochemistry.

It is sensible to go back through a story and cut out every paragraph and every sentence where the writer has intruded on the narration to drop in a tidbit about the world in which the characters are interacting. That’s not worldbuilding. Worldbuilding is the responsibility the writer has to create a context that hangs together from the beginning of the work to the end. Your complaint seems aimed at the immature writer (and David Weber) who can’t restrain himself from dropping his research into his book for the sake of wordcount.

\item \textbf{orfanum}

October 1, 2007 at 5:57 pm

I donno really - what would you say of The Gormenghast Trilogy? That seems to me to do an incredibly good job of building a world (social as much as physical) almost in the despised ‘Mythopoeic’ manner but by being exactly specific and particular (I recall some of the visual angles and views that Peake forces us to adopt that are minutely concrete aspects of particularity).

Is this though a contempt for the three-decker fantasy that has a socio-political origin disguised rather as an aesthetic distaste? I know Moorcock has panned Tolkien et al for the first reason (i.e., being reactionary, and therefore ruralist, and therefore reactionary, etc.), at least, for example, and I am just wondering whether the creator of such as Swinburne Sinclair-Peter is in the same boat?

(this from an ex-Tolkien reader, by the way, who can no longer stomach it, either)

\item \textbf{uzwi}

October 1, 2007 at 6:25 pm

Quite right that there’s a “sociopolitical” component. But it’s to do with the the creation of addictively immersive texts, as an aspect of the fantasy world we now try to live in. For me, the key (both economic \& aesthetic) lies in those offensive usages “a world” (as opposed to “the world”) \& “secondary world”, with their implication that intricately-invented \& massively realised spaces share something of, or can even replace, reality. The attempt to replace the real isn’t an act of fiction, it’s an act of voodoo. Postmodernism was partly to blame, \& I don’t think it’s a coincidence that postmodernism, neoconservatism \& worldbuilding fantasy went through their explosive growth spurt together.

\item \textbf{orfanum}

October 1, 2007 at 6:56 pm

I don’t know a lot about postmodernism, except that it’s perhaps more mumbo-jumbo than voodoo. Interestingly though afaik it does say quite a bit about the role of the reader, and their freedom, in contradistinction to the power, traditionally, of the author, to trammel and constrain (well, that’s the Idiot’s take on it). That description may lack in accuracy but how does it square with the sentiment you have described above that “Worldbuilding numbs the reader’s ability to fulfil their part of the bargain”.

On the other hand, we all know what Tolkien was about, not so much an escape from reality (although he wrote a lot in order to try to rebuff the escapism charge) as its inversion, its displacement by that other, perfect reality, Heaven - minutely detailed and fixed for all time, immutable, each Angel’s wing unruffled for eternity.

Is this it then - the question of movement and change in opposition to the unalterable? In the Centauri Device John Truck (vroom, vroom) moves from place to place with incredible pace, albeit between somewhat fixed nodes, so that the story is one of movement and transformation; whereas Tolkien’s is not? And this difference is the true centre of the complaint?

\item \textbf{uzwi}

October 1, 2007 at 7:07 pm

Maybe. What’s certain is that the “immobility” of Tolkien’s worlds make them highly susceptible to commodification \& fetishisation. That’s one reason why Tolkien may have inadvertently been a pioneer of worldbuilding fantasy, which was to become the dominant mode in the 1990s.

\end{itemize}

\end{document}
